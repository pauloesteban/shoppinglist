\documentclass [11pt] {article}

\usepackage[spanish]{babel}
\usepackage{graphicx}
\usepackage[utf8]{inputenc} % acentos sin codigo
\usepackage{enumerate} % enumerados

\begin{document}


\begin{figure}[h]
\begin{center} 
 \includegraphics[width=40pt]{C:/descarga.png}UNIVERSIDAD TÉCNICA ESTATAL DE QUEVEDO  \hspace{0,1cm}\includegraphics[width=40pt]{C:/logo_fci.jpg}
\end{center}
\end{figure}

\begin{center}
\bf FACULTAD DE CIENCIAS DE LA INGENIERÍA\\
\bf ESCUELA DE ELÉCTRICA\\

\vspace{0,5cm}
 \bf CARRERA:\\
INGENIERÍA EN TELEMÁTICA

\vspace{0,5cm}
 \bf UNIDAD DE APRENDIZAJE:\\
INTERACCIÓN HOMBRE-MÁQUINA

\vspace{0,5cm}
 \bf TEMA:\\
PROTOTIPO DE BAJA  FIDELIDAD LISTA DE COMPRAS\\
\end{center}

\begin{figure}[h]
\begin{center} 
\includegraphics[width=80pt]{C:/images.jpg}
\end{center}
\end{figure}

\begin{center}
 \bf AUTORES:\\
BRIONES BRAVO JHON KEVIN\\
ESPINOZA MARTÍNEZ ÁLVARO DANIEL\\
SOLÓRZANO CASTRO LUIS ENRIQUE

\vspace{0,5cm}
 \bf DOCENTE:\\
ING. PAULO CHILIGUANO

\vspace{0,5cm}
 \bf PERIÓDO ACADÉMICO\\
2016 – 2017 

\vspace{0,5cm}
 \bf DESCRIPCIÓN\\
\end{center}


Nuestra aplicación comienza con la presentación de una imagen que describe las compras por medio de dispositivos móviles o pc’s, después de determinados segundos inicia y tenemos la pantalla de registrarse o si el usuario ya está registrado inicie la sesión con respectivo mail o contraseña y le damos a la opción login, una vez iniciada la sesión, vamos a otra pantalla donde nos muestra una interfaz para elegir productos tomando las opciones que ofrece como: productos de ferretería, lácteos, carnes, etc. 
\\
\newline
Al momento de que el usuario elija un producto para añadir al carrito de compras, automáticamente se inicia una pequeña ventana para crear la nueva lista de compras, añadir el nombre y aceptar caso contrario cancelar. El usuario puede seguir añadiendo productos al carrito y cuando ya lo haya hecho puede dirigirse a la opción de productos que posee su carrito de compras en la parte superior derecha.
\\
\newline
Una vez allí, se muestra otra pantalla donde detalladamente se podrán ver nombre del producto, cantidad que hayamos añadido y el valor unitario, esto se va a mostrar en la parte derecha ''Detalles \& Precios'', y al final de la misma nos informa el valor total de los productos, si el usuario desea quitar un producto del carrito en la misma pantalla cada producto cuenta con la opción quitar X para eliminar el producto que no desea.
\\

\begin{center}
 \bf RECOMENDACIONES \\
\end{center}

A partir de las encuestas realizadas para la mejora de nuestra aplicación obtuvimos:\\
\begin{itemize}
\item Presentar una imagen al ingresar con el nombre de la aplicación.\\
\item Que cada lista de compra nueva se guarde de manera automática para que la misma pueda ser utilizada nuevamente.\\
\item Opción de eliminar o quitar un producto no deseado en el carrito de compras.\\
\item Aplicación disponible en idioma ingles para personas extranjeras.\\
\end{itemize}


\end{document}

